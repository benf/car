%% Erläuterungen zu den Befehlen erfolgen unter
%% diesem Beispiel.
\documentclass{scrartcl}
\usepackage[utf8]{inputenc}
\usepackage[T1]{fontenc}
\usepackage[ngerman]{babel}
\usepackage{amsmath}
 
\title{Projekt Dokumentation}
\author{Benjamin Franzke, Jan Klemkow und Maik Rungberg}
\date{03. Januar 2010}

\begin{document}

\maketitle
\tableofcontents

\section{Einleitung} %% ToDo
 
Hier kommt die Einleitung. Ihre Überschrift kommt
automatisch in das Inhaltsverzeichnis.

\section{Kommuniation mit dem Computer} %% ToDo

\section{Funkverbingung zwischen Autokontroler und Steuerkontroler} %% ToDo

\section{Quelltext} %% ToDo
	In diesem Kapitel werden die Einzelnen Quelltexte zu dem Projekt erläuert.

\end{document}

%% vim: set sts=0 fenc=utf-8: 
