%% Erläuterungen zu den Befehlen erfolgen unter
%% diesem Beispiel.
\documentclass{scrartcl}
\usepackage[utf8]{inputenc}
\usepackage[T1]{fontenc}
\usepackage[ngerman]{babel}
\usepackage{amsmath}
 
\title{Projekt Dokumentation}
\author{Benjamin Franzke, Jan Klemkow und Maik Rungberg}
\date{03. Januar 2010}

\begin{document}

\maketitle
\tableofcontents

\section{Einleitung} %% TODO
	Diese Projekt-Dokumentation beschreibt die Komplette Durchführung des Mikroprozessortechnik-Projektes,
	den Aufbau der einzelnen Schaltungen und einige Technische Erläuterungen.

	In den folgenden Kapiteln werden die zwei Schaltungen und zusätzliche Entwicklungen und Umbauten erläutert.
	Als Fahrzeugschaltung wird die Schaltung bezeichnet die in das Fahrzeug verbaut wurde.
	Die Steuerschaltung ist am Computer angeschlossen und sendet die Steuerinformationen an die Fahrzeugschaltung.

\section{Umbau am Auto} %% TODO
	Als Grundlage wurde ein Funkferngesteuertes Auto benutzt,
	welches im Internet bestellt wurde (siehe link).
	%% TODO: Bezeichnung des Auto und/oder Link zum Webshop
	Die vorhandene Elektronik zur Funkfernsteuerung wurde entfernt und durch eine eigene ersetzt.
	Vom Auto wurden das Gestell, die Elektromotoren, zum Antrieb und zur Lenkung,
	sowie das Gehäuse und der Akkumolator übernommen.
	An die Kabel der der Motoren wurden Verlängerungen gelöttet,
	welche sich einfacher auf dem Steckbrettern befestigen lassen.

\section{Fahrzeugschaltung} %% TODO
	Die Fahrzeugschaltung besteht im Zentralen aus dem ATMEGA16 und eine Funkemfänger.
	%% TODO: Bezeichnung des Funkemfängers herreussuchen.
	Die Schaltung wird über die im Fahrzeug integrieten 10 Volt Spannungsversorungs mit versort.
	Da der ATMEGA 16 eine Spannungsverogung von 5 Volt benötigt ist ein Festspannungsregler
	%% TODO: Typ des Festspannungsreglers im Auto
	zwischen Akkumolator und Microkontroler geschaltet.

	\subsection{Motoransteuerung}
		Für die Ansteuerung der Elektromotoren des Antriebs und der Lenkung sind vier Motortreiber, vom Typ L298, verbaut.
		Diese sind notwendig, um den für den Mikrokontroler zu hohen Strom für die Motoren zu regulieren.
		Die Geschwindigkeitsregelung ist mittels Pulsweitenmodulation im Microcontroler implementiert.
		Die verwendeten Bauelemente enthalten zwei Motortreiber. Insgesammt werden zwei Bauelemente und damit vier Motortreiber benutzt.
		Drei Motortreiber sind für den Antriebsmotor parallel zusammen geschaltet.
		Diese ist notwendig um den hohen Stromfluss für den Antrieb zu gewährleisten.
		
		%% TODO: Hier ein Bild vom internen Aufbau des Motortreibers einbinden.
	
	\subsection{Programmierung} %% TODO: Erläuterung der Programmierung der Fahrzeugschaltung
		In diesem Abschnitt wird der Programmcode des Steuerkontrolers erklärt.

\section{Steuerschaltung}

	\subsection{Kommuniation mit dem Computer} %% TODO
		Die Kommunikation zwischen Steuerschaltung und Computer findet über RS-232 statt.
		...
		Für den notwendigen 12 Volt-Pegel wird der MAX232 benutzt.
		...	

	\subsection{Anbindung des Gamepad} %% TODO
		Ein Gamepad wird für die Schnittstelle zum Benutzer verwendet.
		Dafür wurde ein Programm in der Programmiersprache C implementiert,
		welches die Steuerinformationen vom Gamepad über die RS-232 Schnittstelle zum Steuerkontroler weiterleitet.
	
	\subsection{Programmierung} %% TODO: Erläuterung der Programmierung der Steuerschaltung

\section{Funkverbingung zwischen Autokontroler und Steuerkontroler} %% TODO

\section{Quelltext} %% TODO
	In diesem Kapitel werden die Einzelnen Quelltexte zu dem Projekt erläuert.

\end{document}

%% vim: set sts=0 fenc=utf-8: 
