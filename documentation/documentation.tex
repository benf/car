%% Erläuterungen zu den Befehlen erfolgen unter
%% diesem Beispiel.
\documentclass{scrartcl}
\usepackage[utf8]{inputenc}
\usepackage[T1]{fontenc}
\usepackage[ngerman]{babel}
\usepackage{amsmath}
 
\title{Projekt Dokumentation}
\author{Benjamin Franzke, Jan Klemkow und Maik Rungberg}
\date{03. Januar 2010}

\begin{document}

\maketitle
\tableofcontents

\section{Einleitung} %% TODO
	Diese Projekt-Dokumentation beschreibt die Komplette durchführung des MTP Projektes.

\section{Umbau am Auto} %% TODO
	Als Grundlage wurde ein Funkferngesteuertes Auto benutzt,
	welches im Internet bestellt wurde (siehe link). %% TOTO: Link zu dem Auto (webschop)
	Die vorhandene Elektronik zur Funkfernsteuerung wurde entfernt und durch eine eigene ersetzt.

	Vom Auto wurden das Gestell, die Elektromotoren zum Antrieb und zur Lenkung sowie das Gehäuse und der Akkumolator übernommen.

\section{Fahrzeugschaltung} %% TOTO
	Die Fahrzeugschaltung besteht im Zentralen aus dem ATMEGA16 und eine Funkemfänger.
	%% TOTO: Bezeichnung des Funkemfängers herreussuchen.
	Die Schaltung wird über die im Fahrzeug integrieten 10 Volt Spannungsversorungs mit versort.
	Da der ATMEGA 16 eine Spannungsverogung von 5 Volt benötigt ist ein Festspannungsregler %% TOTO:Typ des Festspannungsreglers im Auto
	zwischen Akkumolator und Microkontroler geschaltet.

	\subsection{Motoransteuerung}
		Für die Ansteuerung der Elektromotoren des Antriebs und der Lenkung sind vier Motortreiber verbaut worden.
		%% TOTO: Motortreiber bezeichung
		Diese sind notwendig um den für den Microcontroler zu hohen Strom für die Motoren zu regilieren.

\section{Kommuniation mit dem Computer} %% TODO
	Die Kommunikation zwischen Steuerschaltung und Computer findet über RS-232 statt.

\section{Funkverbingung zwischen Autokontroler und Steuerkontroler} %% TODO

\section{Quelltext} %% TODO
	In diesem Kapitel werden die Einzelnen Quelltexte zu dem Projekt erläuert.

\end{document}

%% vim: set sts=0 fenc=utf-8: 
