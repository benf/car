%% Erläuterungen zu den Befehlen erfolgen unter
%% diesem Beispiel.
\documentclass{scrartcl}
\usepackage[utf8]{inputenc}
\usepackage[T1]{fontenc}
\usepackage[ngerman]{babel}
\usepackage{amsmath}
 
\title{Projekt Dokumentation}
\author{Benjamin Franzke, Jan Klemkow und Maik Rungberg}
\date{03. Januar 2010}

\begin{document}

\maketitle
\tableofcontents

\section{Einleitung} %% TODO
	Diese Projekt-Dokumentation beschreibt die Komplette durchführung des MTP Projektes.

\section{Umbau am Auto} %% TODO
	Als Grundlage wurde ein Funkferngesteuertes Auto benutzt,
	welches im Internet bestellt wurde (siehe link). %% TOTO: Link zu dem Auto (webschop)
	Die vorhandene Elektronik zur Funkfernsteuerung wurde entfernt und durch eine eigene ersetzt.

	Vom Auto wurden das Gestell, die Elektromotoren zum Antrieb und zur Lenkung sowie das Gehäuse übernommen.

\section{Kommuniation mit dem Computer} %% TODO

\section{Funkverbingung zwischen Autokontroler und Steuerkontroler} %% TODO

\section{Quelltext} %% TODO
	In diesem Kapitel werden die Einzelnen Quelltexte zu dem Projekt erläuert.

\end{document}

%% vim: set sts=0 fenc=utf-8: 
