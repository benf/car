%% Erl�uterungen zu den Befehlen erfolgen unter
%% diesem Beispiel.
\documentclass{scrartcl}
\usepackage[latin1]{inputenc}
\usepackage[T1]{fontenc}
\usepackage[ngerman]{babel}
\usepackage{amsmath}
 
\title{Projekt Dokumentation}
\author{Benjamin Franzke, Jan Klemkow und Maik Rungberg}
\date{03. Januar 2010}

\begin{document}

\maketitle
\tableofcontents

\section{Einleitung}
 
Hier kommt die Einleitung. Ihre �berschrift kommt
automatisch in das Inhaltsverzeichnis.
 
\subsection{Formeln}
  
	\LaTeX{} ist auch ohne Formeln sehr n�tzlich und
	einfach zu verwenden. Grafiken, Tabellen,
	Querverweise aller Art, Literatur- und
	Stichwortverzeichnis sind kein Problem.

	Formeln sind etwas schwieriger, dennoch hier ein
	einfaches Beispiel.  Zwei von Einsteins
	ber�hmtesten Formeln lauten:
	\begin{align}
	E &= mc^2                                  \\
	m &= \frac{m_0}{\sqrt{1-\frac{v^2}{c^2}}}
	\end{align}
	Aber wer keine Formeln schreibt, braucht sich
	damit auch nicht zu besch�ftigen.

\section{Quelltext}
	In diesem Kapitel werden die Einzelnen Quelltexte zu dem Projekt erl�uert.

\end{document}

